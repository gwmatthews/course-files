% Options for packages loaded elsewhere
\PassOptionsToPackage{unicode}{hyperref}
\PassOptionsToPackage{hyphens}{url}
%
\documentclass[
]{article}
\usepackage{lmodern}
\usepackage{amssymb,amsmath}
\usepackage{ifxetex,ifluatex}
\ifnum 0\ifxetex 1\fi\ifluatex 1\fi=0 % if pdftex
  \usepackage[T1]{fontenc}
  \usepackage[utf8]{inputenc}
  \usepackage{textcomp} % provide euro and other symbols
\else % if luatex or xetex
  \usepackage{unicode-math}
  \defaultfontfeatures{Scale=MatchLowercase}
  \defaultfontfeatures[\rmfamily]{Ligatures=TeX,Scale=1}
\fi
% Use upquote if available, for straight quotes in verbatim environments
\IfFileExists{upquote.sty}{\usepackage{upquote}}{}
\IfFileExists{microtype.sty}{% use microtype if available
  \usepackage[]{microtype}
  \UseMicrotypeSet[protrusion]{basicmath} % disable protrusion for tt fonts
}{}
\makeatletter
\@ifundefined{KOMAClassName}{% if non-KOMA class
  \IfFileExists{parskip.sty}{%
    \usepackage{parskip}
  }{% else
    \setlength{\parindent}{0pt}
    \setlength{\parskip}{6pt plus 2pt minus 1pt}}
}{% if KOMA class
  \KOMAoptions{parskip=half}}
\makeatother
\usepackage{xcolor}
\IfFileExists{xurl.sty}{\usepackage{xurl}}{} % add URL line breaks if available
\IfFileExists{bookmark.sty}{\usepackage{bookmark}}{\usepackage{hyperref}}
\hypersetup{
  pdftitle={PYDI 2710: Science or Superstition},
  pdfauthor={George W. Matthews},
  hidelinks,
  pdfcreator={LaTeX via pandoc}}
\urlstyle{same} % disable monospaced font for URLs
\usepackage[margin=1in]{geometry}
\usepackage{longtable,booktabs}
% Correct order of tables after \paragraph or \subparagraph
\usepackage{etoolbox}
\makeatletter
\patchcmd\longtable{\par}{\if@noskipsec\mbox{}\fi\par}{}{}
\makeatother
% Allow footnotes in longtable head/foot
\IfFileExists{footnotehyper.sty}{\usepackage{footnotehyper}}{\usepackage{footnote}}
\makesavenoteenv{longtable}
\usepackage{graphicx,grffile}
\makeatletter
\def\maxwidth{\ifdim\Gin@nat@width>\linewidth\linewidth\else\Gin@nat@width\fi}
\def\maxheight{\ifdim\Gin@nat@height>\textheight\textheight\else\Gin@nat@height\fi}
\makeatother
% Scale images if necessary, so that they will not overflow the page
% margins by default, and it is still possible to overwrite the defaults
% using explicit options in \includegraphics[width, height, ...]{}
\setkeys{Gin}{width=\maxwidth,height=\maxheight,keepaspectratio}
% Set default figure placement to htbp
\makeatletter
\def\fps@figure{htbp}
\makeatother
\setlength{\emergencystretch}{3em} % prevent overfull lines
\providecommand{\tightlist}{%
  \setlength{\itemsep}{0pt}\setlength{\parskip}{0pt}}
\setcounter{secnumdepth}{-\maxdimen} % remove section numbering
\usepackage{nopageno}
\usepackage{tcolorbox}
\usepackage{setspace}
\usepackage{fancyhdr}
\pagestyle{fancy}
\lhead{Plymouth State University}
\chead{}
\rhead{Course Syllabus}
\lfoot{}
\cfoot{}
\rfoot{}
\renewcommand{\headrulewidth}{0.5pt}
\renewcommand{\footrulewidth}{0pt}
\newcommand{\wide}{\setlength{\parskip}{2ex}}
%\hypersetup{draft}
\hypersetup{bookmarks=false}
\onehalfspacing



\newtcolorbox{note}{
center,
  colback=black!5!white,
  colframe=black!30,
  coltext=black,
  text width=16cm,
  boxsep=5pt,
  arc=4pt}

\newtcolorbox{caution}{
center,
  colback=black!5!white,
  colframe=black!30,
  coltext=black,
  text width=16cm,
  boxsep=5pt,
  arc=4pt}
  
\newtcolorbox{groups}{
  center,
  colback=black!5!white,
  colframe=black!30,
  coltext=black,
  text width=16cm,
  boxsep=5pt,
  arc=4pt}
  
\newtcolorbox{assignment}{
  center,
  colback=black!5!white,
  colframe=black!30,
  coltext=black,
  text width=16cm,
  boxsep=5pt,
  arc=4pt}

\newtcolorbox{infobox}{
  center,
  colback=black!5!white,
  colframe=black!30,
  coltext=black,
  text width=16cm,
  boxsep=5pt,
  arc=4pt}

\title{PYDI 2710: Science or Superstition}
\author{George W. Matthews}
\date{Fall 2020}

\begin{document}
\maketitle

\thispagestyle{fancy}

\begin{center}\rule{0.5\linewidth}{0.5pt}\end{center}

\hypertarget{relativism}{%
\subsection{\texorpdfstring{\emph{Relativism}}{Relativism}}\label{relativism}}

\begin{card-row}

\begin{small-card}

There is no such thing as absolute values.

\end{small-card}

\end{card-row}

\begin{center}\rule{0.5\linewidth}{0.5pt}\end{center}

\hypertarget{online-schedule}{%
\subsection{\texorpdfstring{\emph{Online
Schedule}}{Online Schedule}}\label{online-schedule}}

\begin{infobox}

\begin{longtable}[]{@{}lcccc@{}}
\toprule
\begin{minipage}[b]{0.12\columnwidth}\raggedright
\strut
\end{minipage} & \begin{minipage}[b]{0.19\columnwidth}\centering
M\strut
\end{minipage} & \begin{minipage}[b]{0.19\columnwidth}\centering
T\strut
\end{minipage} & \begin{minipage}[b]{0.16\columnwidth}\centering
W\strut
\end{minipage} & \begin{minipage}[b]{0.19\columnwidth}\centering
TH\strut
\end{minipage}\tabularnewline
\midrule
\endhead
\begin{minipage}[t]{0.12\columnwidth}\raggedright
Week 1\strut
\end{minipage} & \begin{minipage}[t]{0.19\columnwidth}\centering
\strut
\end{minipage} & \begin{minipage}[t]{0.19\columnwidth}\centering
\textbf{video}\\
11 AM - noon\\
2 PM - 3 PM\strut
\end{minipage} & \begin{minipage}[t]{0.16\columnwidth}\centering
\strut
\end{minipage} & \begin{minipage}[t]{0.19\columnwidth}\centering
\textbf{video}\\
11 AM - noon\\
2 PM - 3 PM\strut
\end{minipage}\tabularnewline
\begin{minipage}[t]{0.12\columnwidth}\raggedright
Week 2\strut
\end{minipage} & \begin{minipage}[t]{0.19\columnwidth}\centering
\textbf{text}\\
7 - 8 PM\\
8 - 9 PM\strut
\end{minipage} & \begin{minipage}[t]{0.19\columnwidth}\centering
\strut
\end{minipage} & \begin{minipage}[t]{0.16\columnwidth}\centering
\textbf{text}\\
7 - 8 PM\\
8 - 9 PM\strut
\end{minipage} & \begin{minipage}[t]{0.19\columnwidth}\centering
\strut
\end{minipage}\tabularnewline
\bottomrule
\end{longtable}

\begin{itemize}
\tightlist
\item
  REPEATS EVERY TWO WEEKS *
\end{itemize}

\end{infobox}

\hypertarget{textbook}{%
\subsection{\texorpdfstring{\emph{Textbook}}{Textbook}}\label{textbook}}

\begin{note}

There is no textbook for this course. Required readings and other
assignments are available at the
\href{https://www.6worlds.net/science-blog/}{Science or Superstition
course blog}. We will be making extensive use of the blog in this course
so please check it out and bookmark it now. In the beginning of the
course there will be a workshop on how to use the blog.

\end{note}

\hypertarget{assignments}{%
\subsection{\texorpdfstring{\emph{Assignments}}{Assignments}}\label{assignments}}

\begin{infobox}

\hypertarget{each-section-of-the-course-lasts-two-weeks-and-covers-one-chapter-in-the-online-text.}{%
\subparagraph{\texorpdfstring{Each \textbf{section} of the course lasts
\textbf{two weeks} and covers one chapter in the online
text.}{Each section of the course lasts two weeks and covers one chapter in the online text.}}\label{each-section-of-the-course-lasts-two-weeks-and-covers-one-chapter-in-the-online-text.}}

\begin{longtable}[]{@{}ccccccc@{}}
\toprule
SECTION & Online & Reading Comments & Blog Post & Blog Comments & Case
Study & Project\tabularnewline
\midrule
\endhead
1. 8/24 - 9/6 & 2 & 4 & & & &\tabularnewline
2. 9/7 - 9/20 & 2 & 4 & 1 & 4 & 1* &\tabularnewline
3. 9/21 - 10/4 & 2 & 4 & 1 & 4 & 2* &\tabularnewline
4. 10/5 - 10/18 & 2 & 4 & 1 & 4 & 3* & begin*\tabularnewline
5. 10/19 - 11/1 & 2 & 4 & & 4 & 4* & work*\tabularnewline
6. 11/2 - 11/20 & 2 & 4 & project* & 4 & 5* & POST*\tabularnewline
\bottomrule
\end{longtable}

\texttt{*\ =\ work\ with\ group}

\textbf{Final Discussion} To conclude the semester we will have an
online ``fishbowl'' discussion on the topic of ``Science or
Superstition?'' This will take place twice on Thurday November 19, at 11
AM and then again at 2 PM.

\end{infobox}

\hypertarget{work-groups}{%
\subsection{Work Groups}\label{work-groups}}

\begin{note}

Everyone in the class will be randomly assigned to a group of 5 (or
maybe 6) students.

\begin{itemize}
\item
  Some assignments -- the case studies and the group project are to be
  done with these groups.*
\item
  Each group has its own dedicated discussion forum that can be used to
  coordinate group work. It is available on the Moodle home page.
\end{itemize}

\end{note}

\hypertarget{assignment-details}{%
\subsection{Assignment Details}\label{assignment-details}}

\begin{assignment}

\hypertarget{online}{%
\subsubsection{Online}\label{online}}

\begin{itemize}
\tightlist
\item
  I'll be online in the video chatroom twice and in the text chatroom
  twice in each section of the course. You have to attend any two of
  these.
\item
  The format of both is open, although at times I'll have specific
  things to talk about. Do expect to participate and come prepared by
  having looked at some recent course material.
\item
  The final text chat of each section will be devoted to the case study
  starting in section 2.
\end{itemize}

\emph{30 points}

\end{assignment}

\begin{assignment}

\hypertarget{reading-comments}{%
\subsubsection{Reading Comments}\label{reading-comments}}

\begin{itemize}
\tightlist
\item
  There are two parts of the blog, the static material that I post on
  content ``pages'' and your always new blog ``posts.'' You will have to
  comment on both.
\item
  Each section you should write 4 comments on any page in the relevant
  reading section. So if it's section 3 comment on any items the third
  section's reading list.
\item
  Please disagree, be clear and write at least 70 words in a comment!
  (this is 70+).
\end{itemize}

\emph{1 point each up to 25 total}

\end{assignment}

\begin{assignment}

\hypertarget{blog-posts}{%
\subsubsection{Blog Posts}\label{blog-posts}}

\begin{itemize}
\tightlist
\item
  Everyone in the course will be required to post three original
  articles to the course blog. Posting deadlines will be determined by
  which group you are in. See
  \href{http://www.6worlds.net/science-blog/science-or-superstition/assignments/}{course
  calendar} for details.
\item
  Read
  \href{https://www.6worlds.net/science-blog/science-or-superstition/chapter-1/1-1/welcome/}{guidelines}
  before posting.
\end{itemize}

\emph{30 points}

\end{assignment}

\begin{assignment}

\hypertarget{blog-comments}{%
\subsubsection{Blog Comments}\label{blog-comments}}

\begin{itemize}
\tightlist
\item
  In addition to posting articles on the blog you will also be required
  to comment \textbf{on your classmates articles} 25 times over the
  course of the semester starting in the second section.
\item
  Again, please disagree and feel free to rsepond to responses.
\end{itemize}

\emph{1 point each up to 20 total}

\end{assignment}

\begin{assignment}

\hypertarget{case-studies}{%
\subsubsection{Case Studies}\label{case-studies}}

\begin{itemize}
\tightlist
\item
  These are short, directed research and reflection activities to be
  done with your groups.
\item
  Text and video chat are easily available to facilitate working as a
  group. We'll set this up in our first online sessions.
\item
  These start in the second section, and are to be turned in as
  submitted Google forms by the end of sections 2-6.
\end{itemize}

\emph{5 points each, 25 points total}

\end{assignment}

\begin{assignment}

\hypertarget{final-project}{%
\subsubsection{Final Project}\label{final-project}}

\begin{itemize}
\tightlist
\item
  This is a collaborative research and presentation project on some area
  of ``Not Quite Science.''
\item
  We'll talk about possible topics, but this will involve some sort of
  visual presentation using some presentation tool such as Prezi,
  Slides, Thinglink, h5P, etc. or photo gallery, art work, video, etc.
\item
  The medium is up to you, but it should be embeddable in a Wordpress
  post (talk to me for technical assistance).
\end{itemize}

\emph{20 points}

\end{assignment}

\begin{assignment}

\hypertarget{final-discussion}{%
\subsubsection{Final Discussion}\label{final-discussion}}

\begin{itemize}
\tightlist
\item
  To conclude the semester we will meet online to settle the question,
  ``Science or superstition?''
\end{itemize}

\emph{10 points}

\end{assignment}

\hypertarget{grades}{%
\subsection{\texorpdfstring{\emph{Grades}}{Grades}}\label{grades}}

\begin{caution}

There are \textbf{160 possible points} in the course.

\begin{itemize}
\tightlist
\item
  \textbf{To pass} you need to earn \textbf{90 points} (55\% of total).
\item
  \textbf{For a C} you need to earn \textbf{110 points} (70\% of total).
\item
  \textbf{For a B} you need to earn \textbf{130 points} (80\% of total).
\item
  \textbf{For an A} you need to earn \textbf{145 points} (90\% of
  total).
\end{itemize}

\end{caution}

\begin{center}\rule{0.5\linewidth}{0.5pt}\end{center}

\hypertarget{accessibility}{%
\subsection{\texorpdfstring{\emph{Accessibility}}{Accessibility}}\label{accessibility}}

\begin{infobox}

Plymouth State University is committed to providing students with
documented disabilities equal access to all university programs and
facilities. If you think you have a disability requiring accommodations,
you should contact Campus Accessibility Services (CAS), located in
Speare (535-3300) to determine whether you are eligible for such
accommodations. According to University policy accommodations will only
be considered for students who have registered with CAS. If you have
authorized CAS to electronically deliver a Letter of Accommodations for
this course, please let me know and we can work out how to meet your
needs.

In addition, I have designed and chosen web-based learning material with
accessibility in mind. If you find any shortcoming in this regard please
let me know and I can fix things.

\end{infobox}

\begin{center}\rule{0.5\linewidth}{0.5pt}\end{center}

\hypertarget{catalog-course-description}{%
\subsubsection{\texorpdfstring{\emph{Catalog Course
Description}}{Catalog Course Description}}\label{catalog-course-description}}

Utilizes scientific methodologies to investigate, analyze, and interpret
data to propose answers, offer explanations, and make predictions to
philosophically analyze the power and limitations of science.
Distinguishes science from irrational opinion and superstition. Explores
the extent to which science is a way to knowledge, and finds
philosophical principles that can guide us in evaluating controversial
beliefs.

\hypertarget{course-learning-outcomes}{%
\subsubsection{\texorpdfstring{\emph{Course learning
outcomes}}{Course learning outcomes}}\label{course-learning-outcomes}}

\begin{enumerate}
\def\labelenumi{\arabic{enumi}.}
\tightlist
\item
  Utilize scientific methods to plan, investigate, collect, analyze, and
  interpret data to propose answers and explanations about reality.
\item
  Engage in significant field or lab work to study and analyze phenomena
  and make predictions about experiences in contemporary environments.
\item
  Identify the power and limitations of scientific methodologies.
\item
  Distinguish between science, pseudoscience, and superstition.
\item
  Identify philosophical principles that can guide us in evaluating
  controversial beliefs.
\item
  Distinguish between scientific reasoning and irrational opinion.
\end{enumerate}

\hypertarget{ada-statement}{%
\subsubsection{\texorpdfstring{\emph{ADA
Statement}}{ADA Statement}}\label{ada-statement}}

Plymouth State University is committed to providing students with
documented disabilities equal access to all university programs and
facilities. If you think you have a disability requiring accommodations,
you should contact Campus Accessibility Services (CAS), located in
Speare (535-3300) to determine whether you are eligible for such
accommodations. Academic accommodations will only be considered for
students who have registered with CAS. If you have authorized CAS to
electronically deliver a Letter of Accommodations for this course,
please communicate with your instructor to review your accommodations.

\end{document}
